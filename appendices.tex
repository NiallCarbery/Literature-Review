% Appendices
\appendix

\chapter{Appendixes}\label{appendix:appendices}

\qquad We include relevant derivations in full detail, as well as additional numerical results for larger systems and random Ising instances. These results further corroborate the findings presented in the main text, and provide a more comprehensive picture of the protocol's performance across different system sizes and problem instances.

%----------------------------------------------------------------------------------

\section{Dephasing Effects Derivation} \label{DephasingEffectsDerivation}

\qquad A useful definition are Kraus operators which offer a way representing any completely positive, trace-preserving (CPTP) map on a quantum system. A generic update for an open system is expressed as $\rho\mapsto\sum_i K_i\rho K_i^{\dagger}$, where the set $\{K_i\}$ encodes the environmental influence or measurement back-action. These maps must satisfy the completeness condition $\sum_i K_i^{\dagger}K_i=\mathds{1}$. In this specific instance, the two non-zero operators capture the effective dynamics remaining after the meter states are traced out.

\qquad We consider a simple example using a qubit meter and show that the protocol described in the text induces dephasing on the system. Suppose the meter is initialised in
%%
\begin{equation}
    |+\rangle = \frac{1}{\sqrt{2}}\big(|0\rangle + |1\rangle\big),
\end{equation}
%%
where $\sigma_z|0\rangle = |0\rangle$ and $\sigma_z|1\rangle = -|1\rangle$. Set $X_M = x_0\sigma_z$, $H_M=0$ and keep $H_S(t)$ general. Here $x_0\in\mathbb{R}$ scales the strength of the QND coupling. The total Hamiltonian is
%%
\begin{equation}\label{eq:Htot_meter}
    H(t) = H_S(t)\otimes\mathds{1}_M + x_0 H_S(t)\otimes\sigma_z
    = (1+x_0)H_S(t)\otimes|0\rangle\langle0| + (1-x_0)H_S(t)\otimes|1\rangle\langle1|.
\end{equation}
%%
The time evolution operator for the joint system can therefore be written below, where $U^{[x]}_{\rm QND}(t)$ is the propagator generated by the rescaled system Hamiltonian $xH_S(t)$and $x=1\pm x_0$.
%%
\begin{align}\label{eq:U_tot}
    U(t) = U^{[1+x_0]}_{\rm QND}(t)\otimes|0\rangle\langle0| + U^{[1-x_0]}_{\rm QND}(t)\otimes|1\rangle\langle1| \nonumber \\
    i\hbar\,\frac{d}{dt}U^{[x]}_{\rm QND}(t) = x H_S(t)\,U^{[x]}_{\rm QND}(t),
\end{align}
%%
Suppose the initial joint state is separable $\rho_S(0)\otimes|+\rangle\langle+|$. Tracing out the meter yields a Kraus decomposition for the system dynamics. Because $|+\rangle$ has support on both meter basis states, the non-zero Kraus operators are
%%
\begin{subequations}
\begin{align}
    K_{++}(t) &= \frac{1}{\sqrt{2}}\Big(U^{[1+x_0]}_{\rm QND}(t) + U^{[1-x_0]}_{\rm QND}(t)\Big), \\
    K_{-+}(t) &= \frac{1}{\sqrt{2}}\Big(U^{[1+x_0]}_{\rm QND}(t) - U^{[1-x_0]}_{\rm QND}(t)\Big).
\end{align}
\end{subequations}
%%
The system density operator at time $t$ is
%%
\begin{equation}\label{eq:rho_kraus}
    \rho_S(t) = K_{++}(t)\,\rho_S(0)\,K_{++}^\dagger(t) + K_{-+}(t)\,\rho_S(0)\,K_{-+}^\dagger(t).
\end{equation}
%%
Expanding and simplifying gives
%%
\begin{equation}\label{eq:rho_halfsum}
    \rho_S(t) = \tfrac{1}{2}U^{[1+x_0]}_{\rm QND}(t)\,\rho_S(0)\,U^{[1+x_0]\dagger}_{\rm QND}(t) + \tfrac{1}{2}U^{[1-x_0]}_{\rm QND}(t)\,\rho_S(0)\,U^{[1-x_0]\dagger}_{\rm QND}(t).
\end{equation}
%%
Thus half of the initial ensemble evolves under the Hamiltonian $(1+x_0)H_S(t)$ while the other half evolves under $(1-x_0)H_S(t)$, and the final state is the equal mixture of these two evolutions. Differentiating yields
%%
\begin{equation}\label{eq:rho_dot_half}
    \frac{d}{dt}\rho_S(t) = \tfrac{1}{2}\frac{d}{dt}\rho^{[1+x_0]}_S(t) + \tfrac{1}{2}\frac{d}{dt}\rho^{[1-x_0]}_S(t),
\end{equation}
%%
where $\rho^{[1\pm x_0]}_S(t)=U^{[1\pm x_0]}_{\rm QND}(t)\,\rho_S(0)\,U^{[1\pm x_0]\dagger}_{\rm QND}(t)$. Here we demonstrate how to express Eq.~\eqref{eq:rho_dot_half} in the instantaneous eigenbasis of $H_S(t)$. For this, we first consider pure states and start from the Schr\"odinger equation for the amplitudes in the instantaneous eigenbasis of $(1 \pm x_0)H_S(t)$, which is evidently the same eigenbasis as $H_S(t)$. With $|\psi(t)\rangle = \sum_m c_m(t) |m(t)\rangle$, the amplitudes evolve according to
%%
\begin{align}
    \frac{d}{dt} c_m^{[1\pm x_0]}(t) =& -i(1 \pm x_0)E_m(t)c_m^{[1\pm x_0]}(t) \nonumber \\
    &- \langle m(t)| \dot{m}(t)\rangle c_m^{[1\pm x_0]}(t) \nonumber \\
    &+ \sum_{n \neq m} \frac{\langle m(t)|(1 \pm x_0) \dot{H}_S(t)|n(t)\rangle}{(1 \pm x_0)(E_m(t) - E_n(t))} c_n^{[1\pm x_0]}(t).
\end{align}
%%
Since $(1 \pm x_0)$ cancels in the last term, it only appears in the first (phase) term. We can use this to get a differential equation for the density matrices via
%%
\begin{align}
    \frac{d}{dt} ((\rho_S^{[1\pm x_0]}(t))_{mn}) &= \frac{d}{dt} ((c_m^{[1\pm x_0]}(t))^* c_n^{[1\pm x_0]}(t)) \nonumber \\
    &= c_n^{[1\pm x_0]}(t) \frac{d}{dt} (c_m^{[1\pm x_0]}(t))^* + (c_m^{[1\pm x_0]}(t))^* \frac{d}{dt} c_n^{[1\pm x_0]}(t) \nonumber \\
    &= i(1 \pm x_0)(E_m(t) - E_n(t))(\rho_S^{[1\pm x_0]}(t))_{mn} - (\langle m(t)| \dot{m}(t)\rangle - \langle n(t)| \dot{n}(t)\rangle)(\rho_S^{[1\pm x_0]}(t))_{mn} \nonumber \\
    &+ \sum_{i \neq m} \frac{\langle i(t)| \dot{H}_S(t)|m(t)\rangle}{E_m(t) - E_i(t)} (\rho_S^{[1\pm x_0]}(t))_{mi} + \sum_{j \neq n} \frac{\langle n(t)| \dot{H}_S(t)|j(t)\rangle}{E_n(t) - E_j(t)} (\rho_S^{[1\pm x_0]}(t))_{jn}.
\end{align}
%%
Inserting this result into Eq.~\eqref{eq:rho_dot_half} and simplifying, we obtain
%%
\begin{align}
    \frac{d}{dt} ((\rho_S(t))_{mn}) =& i(E_m(t) - E_n(t)) \frac{1}{2} ((\rho_S^{[1+x_0]}(t))_{mn} + (\rho_S^{[1-x_0]}(t))_{mn}) \nonumber \\
    &- (\langle m(t)| \dot{m}(t)\rangle - \langle n(t)| \dot{n}(t)\rangle) \frac{1}{2} ((\rho_S^{[1+x_0]}(t))_{mn} + (\rho_S^{[1-x_0]}(t))_{mn}) \nonumber \\
    &+ \sum_{i \neq m} \frac{\langle i(t)| \dot{H}_S(t)|m(t)\rangle}{E_m(t) - E_i(t)} \frac{1}{2} ((\rho_S^{[1+x_0]}(t))_{mi} + (\rho_S^{[1-x_0]}(t))_{mi}) \nonumber \\
    &+ \sum_{j \neq n} \frac{\langle n(t)| \dot{H}_S(t)|j(t)\rangle}{E_n(t) - E_j(t)} \frac{1}{2} ((\rho_S^{[1+x_0]}(t))_{jn} + (\rho_S^{[1-x_0]}(t))_{jn}) \nonumber \\
    &+ i\frac{x_0}{2} (E_m(t) - E_n(t)) ((\rho_S^{[1+x_0]}(t))_{mn} - (\rho_S^{[1-x_0]}(t))_{mn}).
\end{align}
%%
The first four lines exactly match the rate of change of a density matrix evolving under $H_S(t)$, with the last line as a correction:
%%
\begin{equation}
    \frac{d}{dt} (\rho_S(t))_{mn} = -i ([H_S(t), \rho_S(t)])_{mn} + i\frac{x_0}{2} (E_m(t) - E_n(t)) ((\rho_S^{[1+x_0]}(t))_{mn} - (\rho_S^{[1-x_0]}(t))_{mn}).
\end{equation}
%%
Although this term implies a modification of coherences, practically it results in suppression since pure states are maximally coherent. This result stems from an exact Kraus decomposition, it remains valid outside the adiabatic limits, applicable even during rapid system dynamics or strong interactions.

%----------------------------------------------------------------------------------

\section{Additional Figures: Larger Ising Systems}

\qquad We have been able to attain results for larger Ising systems up to $N=7$ qubits, which further corroborate the findings presented in the main text. These have been accomplished on standard laptop hardware by optimizing numerical evolution in C and Cython\cite{qutip5} and parallelizing over multiple random instances. The results are presented in the following figures, which are variants of Figs. \ref{fig:fidelity_contour} and \ref{fig:fidelity_diff} from the main text.

\begin{figure}[H]
    \centering
    \begin{subfigure}[t]{0.32\textwidth}
        \centering
        \refstepcounter{subfigure}\makebox[\textwidth][l]{\hspace{-2mm}(\thesubfigure)}\label{fig:larger_isings_2b_a}
        \vspace{-1.2ex}
        \includegraphics[width=\textwidth]{LargerIsings/2-2b.png}
    \end{subfigure}%
    \hfill
    \begin{subfigure}[t]{0.32\textwidth}
        \centering
        \refstepcounter{subfigure}\makebox[\textwidth][l]{\hspace{-2mm}(\thesubfigure)}\label{fig:larger_isings_2b_b}
        \vspace{-1.2ex}
        \includegraphics[width=\textwidth]{LargerIsings/3-2b.png}
    \end{subfigure}%
    \hfill
    \begin{subfigure}[t]{0.32\textwidth}
        \centering
        \refstepcounter{subfigure}\makebox[\textwidth][l]{\hspace{-2mm}(\thesubfigure)}\label{fig:larger_isings_2b_c}
        \vspace{-1.2ex}
        \includegraphics[width=\textwidth]{LargerIsings/4-2b.png}
    \end{subfigure}

    \vspace{1ex}

    \begin{subfigure}[t]{0.32\textwidth}
        \centering
        \refstepcounter{subfigure}\makebox[\textwidth][l]{\hspace{-2mm}(\thesubfigure)}\label{fig:larger_isings_5_2b}
        \vspace{-1.2ex}
        \includegraphics[width=\textwidth]{LargerIsings/5-2b.png}
    \end{subfigure}
    \caption[Larger Ising systems variant of Fig. \ref{fig:fidelity_contour}]{Larger Ising systems variant of Fig. \ref{fig:fidelity_contour}. (a) 2-qubit, (b) 3-qubit, (c) 4-qubit, (d) 5-qubit systems.}
    \label{fig:larger_isings_2b}
\end{figure}

\qquad Invariant fidelity along contours of $T_eff = T(1+x0)$ holds with increasing system size, fidelity decrease with system size as expected as energy gaps become smaller for more complex systems.

\begin{figure}[htbp]
    \centering
    % First row: 3 panels (N=3,4,5)
    \begin{subfigure}[t]{0.32\textwidth}
        \centering
        \refstepcounter{subfigure}\makebox[\textwidth][l]{\hspace{-2mm}(\thesubfigure)}\label{fig:larger_isings_4b_a}
        \vspace{-1.2ex}
        \includegraphics[width=\textwidth]{LargerIsings/3-4b.png}
    \end{subfigure}%
    \hfill
    \begin{subfigure}[t]{0.32\textwidth}
        \centering
        \refstepcounter{subfigure}\makebox[\textwidth][l]{\hspace{-2mm}(\thesubfigure)}\label{fig:larger_isings_4b_b}
        \vspace{-1.2ex}
        \includegraphics[width=\textwidth]{LargerIsings/4-4b.png}
    \end{subfigure}%
    \hfill
    \begin{subfigure}[t]{0.32\textwidth}
        \centering
        \refstepcounter{subfigure}\makebox[\textwidth][l]{\hspace{-2mm}(\thesubfigure)}\label{fig:larger_isings_4b_c}
        \vspace{-1.2ex}
        \includegraphics[width=\textwidth]{LargerIsings/5-4b.png}
    \end{subfigure}

    \vspace{1ex}

    % Second row: 2 panels (N=6,7)
    \begin{subfigure}[t]{0.32\textwidth}
        \centering
        \refstepcounter{subfigure}\makebox[\textwidth][l]{\hspace{-2mm}(\thesubfigure)}\label{fig:larger_isings_6_7_a}
        \vspace{-1.2ex}
        \includegraphics[width=0.95\textwidth]{LargerIsings/6-4b.png}
    \end{subfigure}%
    \hfill
    \begin{subfigure}[t]{0.32\textwidth}
        \centering
        \refstepcounter{subfigure}\makebox[\textwidth][l]{\hspace{-2mm}(\thesubfigure)}\label{fig:larger_isings_6_7_b}
        \vspace{-1.2ex}
        \includegraphics[width=0.95\textwidth]{LargerIsings/7-4b.png}
    \end{subfigure}
        \hfill
    \begin{subfigure}[t]{0.32\textwidth}
        \centering
        \refstepcounter{subfigure}\makebox[\textwidth][l]{\hspace{-2mm}(\thesubfigure)}\label{fig:larger_isings_6_7_b}
        \vspace{-1.2ex}
        \includegraphics[width=0.95\textwidth]{LargerIsings/7-4b.png}
    \end{subfigure}

    \caption[Results for larger systems (N=3--7)]{Larger Ising systems variants of Fig. \ref{fig:fidelity_diff}. (a) 3-qubit, (b) 4-qubit, (c) 5-qubit, (d) 6-qubit, (e) 7-qubit systems.}
    \label{fig:larger_isings_4b}
\end{figure}

\qquad Internal meter dynamics reduce speed up relative to coherent case, interestingly increasing system size appears to reduce this effect. Most likely due to larger systems suffering more from non-adiabatic effects, thus the relative contribution of internal meter dynamics is reduced. This is an interesting effect that warrants further investigation, and may suggest that meter coupling could be more effective for larger systems where non-adiabatic effects are more pronounced.

\begin{figure}[htbp]
    \centering
    \begin{subfigure}[t]{0.48\textwidth}
        \centering
        \refstepcounter{subfigure}\makebox[\textwidth][l]{\hspace{-2mm}(\thesubfigure)}\label{fig:larger_systems_fidelity_a}
        \vspace{-1.2ex}
        \includegraphics[width=0.95\textwidth]{LargerIsings/FidelityIncrease.png}
    \end{subfigure}%
    \hfill
    \begin{subfigure}[t]{0.48\textwidth}
        \centering
        \refstepcounter{subfigure}\makebox[\textwidth][l]{\hspace{-2mm}(\thesubfigure)}\label{fig:larger_systems_fidelity_b}
        \vspace{-1.2ex}
        \includegraphics[width=0.95\textwidth]{LargerIsings/FidelityIncreaseMesh.png}
    \end{subfigure}
    \caption[Fidelity increase for larger systems]{\textbf{Possible Graphing for Larger Systems need more Data points}. (a) Fidelity increase, (b) Fidelity increase mesh.}
    \label{fig:larger_systems_fidelity}
\end{figure}
%%
\qquad 
\newpage

%----------------------------------------------------------------------------------

\section{Annealing Gap Profiles for Random Ising Instances}\label{sec:appendix_gap_statistics}
\qquad We present the energy gap statistics for the $100,000$ random Ising instances. These could be used to generate less crude annealing schedules however our experience is they lead in minimal improvement over the average optimal schedule, and thus we have not included them in the main text. Large gap statistics would need to be generated to extract trends for minimum gap location and size. However, these are highly Ising / optimization problem dependent. 
\begin{table}[htbp]
    \centering
    \begin{tabular}{lccc}
        \hline
        Metric & $N=3$ & $N=4$ & $N=5$ \\
        \hline
        Number of samples & 100,000 & 100,000 & 100,000 \\
        Global minimum gap & 0.000004 & 0.000003 & 0.000007 \\
        Mean minimum gap & 0.387 & 0.330 & 0.315 \\
        Std of minimum gap & 0.247 & 0.227 &  0.215 \\
        Mean min gap location ($s$) & 0.779 & 0.811 & 0.739 \\
        Std of min gap location & 0.142 & 0.153 & 0.155 \\
        Coeff. of variation at min & 0.599 & 0.647 & 0.613 \\
        10th percentile minimum & 0.080 & 0.064 & 0.065 \\
        25th percentile minimum & 0.203 & 0.168 & 0.175 \\
        \hline
    \end{tabular}
    \caption{Summary statistics for energy gaps in random Ising spin chains ($J_{ij}\in[0,1]$) for $N=3,4,5$ qubits, averaged over 100,000 samples.}
\end{table}
\newpage
\section{Average Annealing Schedules for different Instances}\label{sec:appendix_larger_annealings}

\qquad Below we extend the average optimal schedules for different random Ising instances with $N=3,4,5$ qubits, each averaged over 100,000 samples. The plots highlight the performance of the average optimal schedule compared to both the linear schedule and the individual optimal schedules.
\begin{figure}[htbp]
    \centering
    % Row 1
    \begin{subfigure}[t]{0.48\textwidth}
        \centering
        \refstepcounter{subfigure}\makebox[\textwidth][l]{\hspace{-2mm}(\thesubfigure)}\label{fig:opt_ramp_3_a}
        \vspace{-1.5ex}% move image up (so caption appears slightly higher)
        \raisebox{0ex}{\includegraphics[width=0.75\textwidth,trim=0 0 0 0,clip]{OptimalRamp/3-QAAverageRampVsLinear.png}}
    \end{subfigure}%
    \hfill
    \begin{subfigure}[t]{0.48\textwidth}
        \centering
        \refstepcounter{subfigure}\makebox[\textwidth][l]{\hspace{-2mm}(\thesubfigure)}\label{fig:opt_ramp_3_b}
        \vspace{-1.5ex}
        \raisebox{0ex}{\includegraphics[width=0.75\textwidth,trim=0 0 0 0,clip]{OptimalRamp/3-QAAverageVsOptimal.png}}
    \end{subfigure}

    \vspace{1ex}

    % Row 2
    \begin{subfigure}[t]{0.48\textwidth}
        \centering
        \refstepcounter{subfigure}\makebox[\textwidth][l]{\hspace{-2mm}(\thesubfigure)}\label{fig:opt_ramp_4_a}
        \vspace{-1.5ex}
        \raisebox{0ex}{\includegraphics[width=0.75\textwidth,trim=0 0 0 0,clip]{OptimalRamp/4-QAAverageVsLinear.png}}
    \end{subfigure}%
    \hfill
    \begin{subfigure}[t]{0.48\textwidth}
        \centering
        \refstepcounter{subfigure}\makebox[\textwidth][l]{\hspace{-2mm}(\thesubfigure)}\label{fig:opt_ramp_4_b}
        \vspace{-1.5ex}
        \raisebox{0ex}{\includegraphics[width=0.75\textwidth,trim=0 0 0 0,clip]{OptimalRamp/4-QAAverageVsOptimal.png}}
    \end{subfigure}

    \vspace{1ex}

    % Row 3
    \begin{subfigure}[t]{0.48\textwidth}
        \centering
        \refstepcounter{subfigure}\makebox[\textwidth][l]{\hspace{-2mm}(\thesubfigure)}\label{fig:opt_ramp_5_a}
        \vspace{-1.5ex}
        \raisebox{0ex}{\includegraphics[width=0.75\textwidth,trim=0 0 0 0,clip]{OptimalRamp/5-QAAverageVsLinear.png}}
    \end{subfigure}%
    \hfill
    \begin{subfigure}[t]{0.48\textwidth}
        \centering
        \refstepcounter{subfigure}\makebox[\textwidth][l]{\hspace{-2mm}(\thesubfigure)}\label{fig:opt_ramp_5_b}
        \vspace{-1.5ex}
        \raisebox{0ex}{\includegraphics[width=0.75\textwidth,trim=0 0 0 0,clip]{OptimalRamp/5-QAAverageVsOptimal.png}}
    \end{subfigure}

    \caption[Average annealing schedules for N=3-5 systems]{Average annealing schedules compared to the linear schedule and the instance-optimal schedule for $N=3$--$5$. Images on the left represent the improvement of the average schedule compared to a linear schedule, while images on the right show the degradation compared to the instance-optimal schedule: (a,b) $N=3$, (c,d) $N=4$, (e,f) $N=5$.}
    \label{fig:optimal_ramp_average}
\end{figure}

\qquad Over these small instances we see that we maintain a constant improvement over the linear schedule except in the very short time regime where diabtic transitions occur regardless of the schedule. In the quasi-static regime of $T_{eff}\approx9$ we maintain an 8\% in fidelity performance gain.

\section{Impact of different coupling regimes} \label{sec:appendix_different_regimes}

\qquad As part of investigating regimes of meter impact we investigate different regimes of coupling strengths. Throughout this thesis we have remained in the ferromagnetic regime i.e. $J_{ij}\in[0,1]$. Here we present results for both antiferromagnetic couplings $J_{ij}\in[-1,0]$ and mixed couplings $J_{ij}\in[-1,1]$. These coupling strengths are allowed in QUBO formulated annealing problems and thus there effect is included. The results are presented in the following figures, which are variants of Figs. \ref{fig:fidelity_contour} and \ref{fig:fidelity_diff} from the main text 

\begin{figure}[htbp]
    \centering
    \begin{subfigure}[t]{0.48\textwidth}
        \centering
        \refstepcounter{subfigure}\makebox[\textwidth][l]{\hspace{-2mm}(\thesubfigure)}\label{fig:antiferromagnetic_a}
        \vspace{-1.2ex}
        \includegraphics[width=0.75\textwidth]{figures/AntiFerromagnetic/Peak0mesh2b.png}
    \end{subfigure}%
    \hfill
    \begin{subfigure}[t]{0.48\textwidth}
        \centering
        \refstepcounter{subfigure}\makebox[\textwidth][l]{\hspace{-2mm}(\thesubfigure)}\label{fig:antiferromagnetic_b}
        \vspace{-1.2ex}
        \includegraphics[width=0.75\textwidth]{figures/AntiFerromagnetic/Peak0Fig4b.png}
    \end{subfigure}

    \vspace{1ex}

    \begin{subfigure}[t]{0.48\textwidth}
        \centering
        \refstepcounter{subfigure}\makebox[\textwidth][l]{\hspace{-2mm}(\thesubfigure)}\label{fig:antiferromagnetic_c}
        \vspace{-1.2ex}
        \includegraphics[width=0.75\textwidth]{figures/AntiFerromagnetic/AntiF-mesh2b.png}
    \end{subfigure}%
    \hfill
    \begin{subfigure}[t]{0.48\textwidth}
        \centering
        \refstepcounter{subfigure}\makebox[\textwidth][l]{\hspace{-2mm}(\thesubfigure)}\label{fig:antiferromagnetic_d}
        \vspace{-1.2ex}
        \includegraphics[width=0.75\textwidth]{figures/AntiFerromagnetic/AntiF-Fig4b.png}
    \end{subfigure}
    \caption[Antiferromagnetic and mixed-coupling regime results]{Antiferromagnetic / mixed-coupling regime results. (a-b) Represents the mixed coupling regime $J_{ij}\in[-1,1]$, (c-d) Represents the antiferromagnetic coupling regime $J_{ij}\in[-1,0]$. (a,c) display increase in fidelity with interaction strength as in Fig.~\ref{fig:fidelity_contour}; (b,d) show fidelity reduction relative to intrinsic meter dynamics as in Fig.~\ref{fig:fidelity_diff}.}
    \label{fig:antiferromagnetic_results}
\end{figure}


\qquad These regimes display that the ferromagnetic regime represents the worst case results for meter coupling and non commuting effects. Both the antiferromagnetic and mixed coupling regimes display significantly higher fidelities across the parameter space. Thus, we the ferromagnetic regime is the most challenging for meter coupling, and thus the results presented in the main text are suggestive of an average lower bound on performance across all coupling regimes. However, specific optimization Ising Hamiltonian may present a more challenging coupling regime, and thus the performance of meter coupling may be more limited for specific problems. This is an interesting avenue for further research, and may suggest that meter coupling could be more effective for certain classes of problems with specific coupling regimes.


\newpage
\chapter{Additional Papers and Reproduced Results}
\quad To investigate a broad range of Shortcuts to Adiabaticity (STA) protocols, we reproduced results from several key papers in the field. We include abrief note on the various methods applied, and the results achieved, we compare the advantages and disadvantages of these methods in comparison to those of the meter coupling used within this thesis. These papers are summarized as follows Counterdiabatic Optimized Local Driving (COLD)\cite{COLD2023}, 
Reservoir Engineering Shortcuts to Adiabaticity\cite{ResEngSTA},

\section{Counterdiabatic Optimized Local Driving}

\qquad Counterdiabatic Optimized Local Driving (COLD) represents a hybrid approach that combines the analytical framework of Local Counterdiabatic Driving (LCD) with numerical quantum optimal control techniques\cite{COLD2023}. The method addresses a fundamental challenge in shortcuts to adiabaticity: achieving fast, high-fidelity state preparation while maintaining experimental accessibility through local control fields. Unlike standard CD protocols that require knowledge of the full instantaneous eigenspectrum and often demand non-local interactions, COLD optimizes the adiabatic path itself to maximize the effectiveness of approximate, experimentally feasible counterdiabatic terms. We use the counterdiabatic driving formalism to build COLD. For a system evolving under a time-dependent Hamiltonian $H_0(t)$, exact CD requires adding a term that suppresses all diabatic transitions

\begin{equation}
    H_{\text{CD}}(t) = H_0(t) + i\hbar\sum_n |\partial_t n(t)\rangle\langle n(t)|,
\end{equation}

where $|n(t)\rangle$ are the instantaneous eigenstates of $H_0(t)$. This can be equivalently written using the adiabatic gauge potential $A_\lambda = i\hbar\sum_n |\partial_\lambda n\rangle\langle n|$ and the adiabatic gauge field $G_\lambda = \partial_\lambda H_0 - i[A_\lambda, H_0]$, yielding:

\begin{equation}
    H_{\text{CD}}(\lambda) = H_0(\lambda) + \dot{\lambda}G_\lambda,
\end{equation}

where $\lambda$ parameterizes the evolution path. The gauge potential satisfies the closed-form equation $[G_\lambda, H_0] = 0$, which generally lacks easy solutions for many-body systems. Local Counterdiabatic Driving uses an approximate gauge potential instead from a restricted set of local operators $\{O_{\text{LCD}}^{(j)}\}$. The ansatz $A_\lambda = \sum_j \alpha_j O_{\text{LCD}}^{(j)}$ is optimised by minimising the Hilbert-Schmidt norm of $G_\lambda$, equivalent to minimising the action:

\begin{equation}
    S = \int d\lambda \, \text{Tr}[G_\lambda^\dagger G_\lambda].
\end{equation}

For real Hamiltonians like Ising models, a natural choice is $O_{\text{LCD}}^{(j)} = \sigma_j^y$, \textbf{Justify choice of LCD operators}. The resulting LCD Hamiltonian

\begin{equation}
    H_{\text{LCD}}(\lambda) = H_0(\lambda) + \dot{\lambda}\sum_j \alpha_j(\lambda) \sigma_j^y.
\end{equation}

\qquad COLD extends this framework by introducing additional control fields to the bare Hamiltonian, creating a family of dynamical paths:

\begin{equation}
    H_\beta(\lambda, \beta) = H_0(\lambda) + f(\lambda, \beta)O_{\text{opt}},
\end{equation}

where $\beta$ represents optimisable control parameters, $f(\lambda, \beta)$ is a parametrised control function satisfying boundary conditions $f(0, \beta) = f(1, \beta) = 0$, and $O_{\text{opt}}$ provides additional degrees of freedom. The complete COLD Hamiltonian becomes:

\begin{equation}
    H_{\text{COLD}}(\lambda, \beta) = H_\beta(\lambda, \beta) + \dot{\lambda}\sum_j \alpha_j(\lambda, \beta) O_{\text{LCD}}^{(j)},
\end{equation}

where the LCD coefficients $\alpha_j(\lambda, \beta)$ are recalculated for each choice of control parameters. The key insight is that by varying $\beta$, one modifies the instantaneous eigenspectrum of $H_\beta$, thereby changing the adiabatic landscape to favour configurations where the local LCD terms provide maximum suppression of diabatic transitions. The optimisation proceeds by minimising a cost function, typically $C(\beta) = 1 - |\langle\psi_T|\psi_f(\beta)\rangle|$, where $|\psi_T\rangle$ is the target ground state and $|\psi_f(\beta)\rangle$ is the final state after evolution under $H_{\text{COLD}}$. Standard optimisation algorithms such as Powell minimisation or dual annealing are employed. Critically, COLD can be combined with advanced optimal control techniques like CRAB (Chopped Randomised Basis), yielding COLD-CRAB, which makes the parameter landscape bigger through randomisation while maintaining the benefits of approximate CD.


\qquad The original paper applies it to quantum annealing protocols, state preparation in Ising models, and the generation of entangled multipartite states. To validate our implementation of COLD and compare it with the engineered dephasing approach discussed in Section 3, we reproduced key results from Ref.\cite{COLD2023} Focusing on quantum annealing protocols and state preparation in Ising models. We present five key figures that illustrate the performance advantages of COLD across different scenarios and system sizes.

\subsection{Two-Spin Annealing Protocol}
\qquad Cold is demonstrated on the two spin quantum annealing problem with Hamiltonian.
\begin{equation}
    H_0(t) = -2J\sigma_1^z\sigma_2^z - h(\sigma_1^z + \sigma_2^z) + 2h\lambda(t)(\sigma_1^x + \sigma_2^x),
\end{equation}
The scaling function used is given by $\lambda(t) = \sin^2(\pi t/2T)$, which satisfies the required boundary conditions and that $J/h=0.5$. The first order LCD terms are given by an ansatz of the adiabatic gauge potential $A_\lambda = \alpha \sum^2_{i=1} \sigma_i^y$


\begin{figure}[H]
    \centering
    \begin{subfigure}[t]{0.48\textwidth}
        \centering
        \refstepcounter{subfigure}\makebox[\textwidth][l]{\hspace{-2mm}(\thesubfigure)}
        \vspace{-1.5ex}
        \raisebox{0ex}{\includegraphics[width=0.75\textwidth,trim=0 0 0 0,clip]{COLD/Fig1a.png}}
    \end{subfigure}%
    \hfill
    \begin{subfigure}[t]{0.48\textwidth}
        \centering
        \refstepcounter{subfigure}\makebox[\textwidth][l]{\hspace{-2mm}(\thesubfigure)}
        \vspace{-1.5ex}
        \raisebox{0ex}{\includegraphics[width=0.75\textwidth,trim=0 0 0 0,clip]{COLD/Fig1b.png}}
    \end{subfigure}
    \caption[COLD optimization for two-spin Hamiltonian]{Optimization of annealing protocol for the two-spin Hamiltonian. (a) fidelity of the final state as a function of total annealing time $T$ for three cases: bare annealing (no LCD), first-order LCD, and combined first-and second-order LCD. }
    \label{fig:COLD_fig1}
\end{figure}

\qquad

\subsection{BPO versus COLD in Five-Spin Systems}

    \begin{figure}[H]
    \centering
    \begin{subfigure}[t]{0.48\textwidth}
        \centering
        \refstepcounter{subfigure}\makebox[\textwidth][l]{\hspace{-2mm}(\thesubfigure)}
        \vspace{-1.5ex}
        \raisebox{0ex}{\includegraphics[width=0.75\textwidth,trim=0 0 0 0,clip]{COLD/Fig2a.png}}
    \end{subfigure}%
    \hfill
    \begin{subfigure}[t]{0.48\textwidth}
        \centering
        \refstepcounter{subfigure}\makebox[\textwidth][l]{\hspace{-2mm}(\thesubfigure)}
        \vspace{-1.5ex}
        \raisebox{0ex}{\includegraphics[width=0.75\textwidth,trim=0 0 0 0,clip]{COLD/Fig2b.png}}
    \end{subfigure}
    \caption{}
    \label{fig:COLD_fig2}
\end{figure}

\qquad


\section{Reservoir Engineering Shortcuts to Adiabaticity}


\newpage
\printbibliography[heading=subbibliography, title={Appendix References}]
