% Chapter 5: Meter Dynamics with Non-Commuting Hamiltonians
\chapter{Quantifying Non-Commuting Meter Dynamics}\label{Ch5:NonCommutingMeter}

\qquad Chapter~\ref{Ch2:SpeedingUpQuantumAnnealing} established that the QND-like meter protocol achieves a linear speedup factor of $(1+m_{\max})$ when two conditions hold: the meter observable $X_M$ commutes with the meter Hamiltonian $H_M$, and the meter is initialized in an eigenstate of $X_M$. Under these conditions the meter remains frozen in its initial state, and the system evolves under a single, time-independent rescaling of its Hamiltonian. A natural question is what happens when this commutation condition is relaxed.

\qquad Any physical auxiliary system will possess internal dynamics that do not commute perfectly with the coupling observable. When $[H_M, X_M] \neq 0$, three effects arise: (i) the meter undergoes coherent precession away from its initial state, (ii) the system becomes entangled with the meter, and (iii) the effective energy rescaling becomes time-dependent rather than constant. This chapter provides a self-contained, step-by-step analytical treatment of these effects for a qubit meter, derives a closed-form expression for the degraded effective rescaling, and validates the predictions against exact numerical simulations.

\qquad This is physically motivated by annealing implementations on ion-trap devices where we can achieve a global coupling of the meter to the system Hamiltonian but more likely at the cost of the meter having its own intrinsic dynamics. We analytically confirm the $\omega =0$ still results in the maximum speedup.

%----------------------------------------------------------------------------------

\section{Joint Hamiltonian and the Role of Non-Commutation}\label{sec:joint_hamiltonian}
%----------------------------------------------------------------------------------

\qquad We begin by writing the full system--meter Hamiltonian from Chapter~\ref{Ch2:SpeedingUpQuantumAnnealing} in its most general form,
%%
\begin{equation}\label{eq:total_hamiltonian_nc}
    H(t) = H_S(t) \otimes \mathds{1}_M + x_0\, H_S(t) \otimes X_M + \mathds{1}_S \otimes H_M,
\end{equation}
%%
where $H_S(t)$ is the time-dependent system Hamiltonian, $X_M$ is the meter observable that couples to the system, $x_0$ is the coupling strength, and $H_M$ governs the free meter dynamics. Throughout this chapter we specialise to a qubit meter with $X_M = \sigma_z, \qquad H_M = \omega\,\sigma_x$ so that $[H_M, X_M] = [\omega\sigma_x, \sigma_z] = -2i\omega\sigma_y \neq 0$ for any $\omega \neq 0$. The parameter $\omega$ therefore quantifies the departure from the ideal commuting regime. The total Hamiltonian becomes
%%
\begin{equation}\label{eq:total_hamiltonian_explicit}
    H_{\text{tot}}(t) = H_S(t) \otimes \mathds{1}_M + x_0\, H_S(t) \otimes \sigma_z + \mathds{1}_S \otimes \omega\,\sigma_x.
\end{equation}
%%
It is instructive to contrast this with the commuting case ($\omega = 0$). When $\omega = 0$, the Hamiltonian is diagonal in the $\sigma_z$ eigenbasis of the meter,
%%
\begin{equation}\label{eq:commuting_decomposition}
    H_{\text{tot}}(t)\big|_{\omega=0} = (1+x_0)\,H_S(t) \otimes |0\rangle\langle 0| + (1-x_0)\,H_S(t) \otimes |1\rangle\langle 1|,
\end{equation}
%%
and the evolution splits into two non-interacting branches, each with a constant rescaling factor $(1 \pm x_0)$. The meter, once initialized in $|0\rangle$, remains there for all time. In contrast, when $\omega \neq 0$, the off-diagonal $\omega\sigma_x$ term mixes $|0\rangle$ and $|1\rangle$ at every instant, continuously coupling the two branches. The Hamiltonian is no longer block-diagonal in the meter basis, and no time-independent basis disentangles the system from the meter.

%----------------------------------------------------------------------------------
\section{Meter Dynamics in Each Energy Sector}\label{sec:meter_dynamics}
%----------------------------------------------------------------------------------

\qquad To develop an analytical picture, we work in the instantaneous eigenbasis of the system Hamiltonian, $H_S(t) = \sum_n E_n(t)\,|n(t)\rangle\langle n(t)|$. Within each energy sector labelled by $n$, the system eigenvalue $E_n(t)$ acts as a parameter, and the meter sees an effective $2 \times 2$ Hamiltonian. We derive this sector Hamiltonian in three steps.

\qquad We act with the projector $|n(t)\rangle\langle n(t)| \otimes \mathds{1}_M$ on Eq.~\eqref{eq:total_hamiltonian_explicit}. Using $H_S(t)|n(t)\rangle = E_n(t)|n(t)\rangle$, the three terms contribute $H_S(t) \otimes \mathds{1}_M \;\rightarrow\; E_n(t)\,\mathds{1}_M$, $x_0\,H_S(t) \otimes \sigma_z \;\rightarrow\; x_0\,E_n(t)\,\sigma_z$, $\mathds{1}_S \otimes \omega\,\sigma_x \;\rightarrow\; \omega\,\sigma_x.$ Adding these yields the effective meter Hamiltonian in sector $n$
%%
\begin{equation}\label{eq:meter_hamiltonian_sector}
    H_{\text{meter}}^{(n)}(t) = E_n(t)\,\mathds{1}_M + x_0\,E_n(t)\,\sigma_z + \omega\,\sigma_x.
\end{equation}
%%
\qquad The identity term $E_n(t)\,\mathds{1}_M$ produces only an overall phase and does not affect the meter's internal dynamics. The non-trivial part is $\tilde{H}_{\text{meter}}^{(n)}(t) = x_0\,E_n(t)\,\sigma_z + \omega\,\sigma_x$. In the basis $\{|0\rangle, |1\rangle\}$ (eigenstates of $\sigma_z$ with eigenvalues $\pm 1$), this matrix reads
%%
\begin{equation}\label{eq:meter_matrix}
    \tilde{H}_{\text{meter}}^{(n)}(t) = \begin{pmatrix} +x_0\,E_n(t) & \omega \\ \omega & -x_0\,E_n(t) \end{pmatrix}.
\end{equation}
%%
\qquad The diagonal entries $\pm x_0 E_n(t)$ are the energy splittings produced by the system meter coupling, and the off-diagonal entry $\omega$ is the transverse field that drives transitions between the meter states. This is precisely the Hamiltonian of a spin $\tfrac{1}{2}$ particle in a time-dependent longitudinal field $x_0 E_n(t)$ and a constant transverse field $\omega$. The eigenvalues of Eq.~\eqref{eq:meter_matrix} follow from the standard formula for a $2 \times 2$ Hermitian matrix
%%
\begin{equation}\label{eq:meter_eigenvalues}
    \lambda_{n,\pm}(t) = \pm\,\Omega_n(t), \qquad \Omega_n(t) \equiv \sqrt{\big(x_0\,E_n(t)\big)^2 + \omega^2}.
\end{equation}
%%
Here $\Omega_n(t)$ is the instantaneous Rabi frequency of the meter in sector $n$. The corresponding normalised eigenstates are
%%
\begin{equation}\label{eq:meter_eigenstates}
    |+_n(t)\rangle = \cos\theta_n(t)\,|0\rangle + \sin\theta_n(t)\,|1\rangle, \qquad
    |-_n(t)\rangle = -\sin\theta_n(t)\,|0\rangle + \cos\theta_n(t)\,|1\rangle,
\end{equation}
%%
where the mixing angle $\theta_n(t)$ satisfies
\begin{equation}\label{eq:mixing_angle}
    \tan\bigl(2\theta_n(t)\bigr) = \frac{\omega}{x_0\,E_n(t)}.
\end{equation}
%%
We can test this of in two limits, for $\omega \to 0$ $\theta_n \to 0$, so $|+_n\rangle \to |0\rangle$ and $|-_n\rangle \to |1\rangle$. The meter eigenstates coincide with the $\sigma_z$ eigenstates, recovering the ideal two-branch decomposition of Eq.~\eqref{eq:commuting_decomposition}. $\omega \gg x_0 E_n$ $\theta_n \to \pi/4$, so both eigenstates become equal-weight superpositions of $|0\rangle$ and $|1\rangle$. The meter's $\sigma_z$ expectation value averages to zero, and the coupling to the system becomes ineffective.

\qquad The mixing angle depends on the system's energy $E_n(t)$ through the ratio $\omega / (x_0 E_n(t))$. This means that different energy levels of the system experience different meter dynamics a feature that has no analogue in the commuting case and is the root cause of system meter entanglement.

%----------------------------------------------------------------------------------
\section{Rabi Oscillations and Meter Polarisation}\label{sec:rabi_oscillations}
%----------------------------------------------------------------------------------

\qquad Having established the effective meter Hamiltonian in each sector, we now solve for the meter's time evolution explicitly. Consider the meter initialised in $|0\rangle$, the $+1$ eigenstate of $\sigma_z$. Within sector $n$, the meter evolves under $\tilde{H}_{\text{meter}}^{(n)}(t)$. To make the calculation tractable, we first treat the case where $E_n(t)$ varies slowly compared to the Rabi frequency $\Omega_n(t)$---the adiabatic-meter (i.e. $E_n(t)$ changes slowly compared to the meter's Rabi frequency $\Omega(t)$)approximation. This is justified when the system's annealing timescale $T$ is much longer than $1/\Omega_n$.

\subsection{Decomposition into Meter Eigenstates}

\qquad The initial meter state $|0\rangle$ is decomposed in the instantaneous eigenbasis of $\tilde{H}_{\text{meter}}^{(n)}(t)$ at $t=0$:
%%
\begin{equation}\label{eq:meter_decomposition}
    |0\rangle = \cos\theta_n(0)\,|+_n(0)\rangle - \sin\theta_n(0)\,|-_n(0)\rangle.
\end{equation}
%%
This follows directly from inverting Eq.~\eqref{eq:meter_eigenstates}. Under the adiabatic-meter approximation, each component tracks the instantaneous eigenstate and acquires a dynamical phase under the eigenvalues $\pm\Omega_n(t)$:
%%
\begin{equation}\label{eq:meter_time_evolution}
    |\chi_n(t)\rangle = \cos\theta_n(0)\,e^{-i\Phi_n^{+}(t)}\,|+_n(t)\rangle - \sin\theta_n(0)\,e^{-i\Phi_n^{-}(t)}\,|-_n(t)\rangle,
\end{equation}

\subsection[Expectation Value of sigma\_z]{Expectation Value of $\sigma_z$}

\qquad The quantity that controls the effective system rescaling is the meter polarisation $\langle\sigma_z\rangle_n(t) = \langle\chi_n(t)|\sigma_z|\chi_n(t)\rangle$. A crucial point is that the amplitudes in Eq.~\eqref{eq:meter_time_evolution} are set by the \emph{initial} mixing angle $\theta_n(0)$, while the matrix elements of $\sigma_z$ in the instantaneous eigenbasis depend on $\theta_n(t)$. Using Eq.~\eqref{eq:meter_eigenstates} at time~$t$,
%%
\begin{equation}\label{eq:sigma_z_in_eigenbasis}
    \langle +_n(t) | \sigma_z | +_n(t) \rangle = \cos 2\theta_n(t) \quad
    \langle -_n(t) | \sigma_z | -_n(t) \rangle = -\cos 2\theta_n(t)  \quad
    \langle +_n(t) | \sigma_z | -_n(t) \rangle = -\sin 2\theta_n(t)
\end{equation}
%%
Substituting Eq.~\eqref{eq:meter_time_evolution} (with coefficients $\cos\theta_n(0)$, $\sin\theta_n(0)$) and using Eq.~\eqref{eq:sigma_z_in_eigenbasis},
%%
\begin{align}\label{eq:sigma_z_expectation_full}
    \langle\sigma_z\rangle_n(t) &= \cos^2\!\theta_n(0)\,\cos 2\theta_n(t) - \sin^2\!\theta_n(0)\,\cos 2\theta_n(t) + \sin 2\theta_n(0)\,\sin 2\theta_n(t)\,\cos\!\big(\Delta\Phi_n(t)\big) \nonumber\\
    &= \cos 2\theta_n(0)\,\cos 2\theta_n(t) + \sin 2\theta_n(0)\,\sin 2\theta_n(t)\,\cos\!\big(\Delta\Phi_n(t)\big),
\end{align}
%%
where the diagonal terms combine via $\cos^2\!\theta_n(0) - \sin^2\!\theta_n(0) = \cos 2\theta_n(0)$, and the cross terms give $2\cos\theta_n(0)\sin\theta_n(0)\sin 2\theta_n(t) = \sin 2\theta_n(0)\sin 2\theta_n(t)$. Now, from Eq.~\eqref{eq:mixing_angle}, the trigonometric identities at any time~$t$ are
%%
\begin{equation}\label{eq:trig_identities}
    \cos 2\theta_n(t) = \frac{x_0\,E_n(t)}{\Omega_n(t)}, \qquad \sin 2\theta_n(t) = \frac{\omega}{\Omega_n(t)},
\end{equation}
%%
with the same expressions at $t=0$ using $E_n(0)$ and $\Omega_n(0) \equiv \sqrt{(x_0 E_n(0))^2 + \omega^2}$. Substituting into Eq.~\eqref{eq:sigma_z_expectation_full} yields the central result for the meter dynamics:
%%
\begin{equation}\label{eq:sigma_z_final}
    \langle\sigma_z\rangle_n(t) = \frac{x_0^2\,E_n(0)\,E_n(t)}{\Omega_n(0)\,\Omega_n(t)} + \frac{\omega^2}{\Omega_n(0)\,\Omega_n(t)}\,\cos\!\left(2\int_0^t \Omega_n(s)\,ds\right),
\end{equation}
%%
with $\Omega_n(t) = \sqrt{(x_0 E_n(t))^2 + \omega^2}$. The first term is the static component, set by the product $\cos 2\theta_n(0)\,\cos 2\theta_n(t)$. Crucially, it depends on \emph{both} the initial energy $E_n(0)$ and the instantaneous energy $E_n(t)$---a feature absent in the constant-energy case. The second term is an oscillatory component at twice the Rabi frequency. Its amplitude $\omega^2/(\Omega_n(0)\Omega_n(t))$ is largest when $\omega$ dominates both the initial and instantaneous coupling energies, and vanishes in the commuting limit $\omega \to 0$.

%%==========================================================================
\section{Oscillatory Interference Effects}\label{sec:interference}
%%==========================================================================

\qquad The time-averaged analysis of Section~\ref{sec:time_averaged_rescaling} captures the dominant effect of non-commuting dynamics but neglects the oscillatory term in Eq.~\eqref{eq:sigma_z_final}. At finite annealing times, the Rabi oscillations are not fully averaged out, and their signature appears as ripple modulations in the fidelity landscape. This section derives an analytical expression for this interference effect, connecting it to the two-branch structure of the Kraus decomposition established in Appendix~\ref{DephasingEffectsDerivation}.

\subsection{Interference Factor for the Landau-Zener Model}

\qquad For the Landau-Zener model (Eq.~\ref{LandauZenerEquation}), the instantaneous energy gap is
%%
\begin{equation}\label{eq:lz_gap}
    \Delta(t) = E_+(t) - E_-(t) = \sqrt{(vt)^2 + g^2},
\end{equation}
%%
so the accumulated phase difference is $\Delta\phi_{01}(T) = \int_0^T \sqrt{(vs)^2 + g^2}\,ds$. When the meter has non-commuting dynamics $H_M = \omega\sigma_x$, the system--meter coupling modifies the relevant Rabi frequency to
%%
\begin{equation}\label{eq:effective_rabi}
    \Omega_{\text{eff}}(t) = \sqrt{E(t)^2 + \omega^2},
\end{equation}
%%
where $E(t) = vt/2$ is the longitudinal energy scale. The interference factor generalises to
%%
\begin{equation}\label{eq:interference_factor_general}
    C(T, \omega) = \cos\!\left(2x_0\int_0^T \Omega_{\text{eff}}(s)\,ds\right) = \cos\!\left(2x_0\int_0^T \sqrt{E(s)^2 + \omega^2}\,ds\right).
\end{equation}
%%
The factor of 2 arises because the full phase accumulation involves both branches of the evolution ($+x_0$ and $-x_0$). The zeros of this interference factor, $C(T, \omega) = 0$, occur when the argument equals $(2n+1)\pi/2$ for integer $n$, defining contours in the $(T, \omega)$ plane along which the two branches completely cancel.

%%==========================================================================
\section{Numerical Validation}\label{sec:numerical_validation}
%%==========================================================================

\qquad We now compare the analytical predictions derived above against exact numerical simulations of the full system--meter Schr\"odinger equation. The simulations use two test systems from Chapter~\ref{Ch2:SpeedingUpQuantumAnnealing}: (i) the Landau-Zener qubit (Eq.~\ref{LandauZenerEquation}) and (ii) a 3-qubit random Ising chain (Eq.~\ref{IsingProblemEquation}). In both cases the meter is a qubit with $X_M = \sigma_z$ and $H_M = \omega\sigma_x$, initialised in $|0\rangle$, with coupling $x_0 = 1$.

\begin{figure}[htbp]
    \centering
    \begin{subfigure}[t]{0.48\textwidth}
        \centering
        \refstepcounter{subfigure}\makebox[\textwidth][l]{\hspace{-2mm}(\thesubfigure)}\label{fig:NonCommutationRescaling_a}
        \vspace{-1.2ex}
            \includegraphics[width=0.95\textwidth]{NonCommutation/SigmaZTimeTrace.png}
    \end{subfigure}%
    \hfill
    \begin{subfigure}[t]{0.48\textwidth}
        \centering
        \refstepcounter{subfigure}\makebox[\textwidth][l]{\hspace{-2mm}(\thesubfigure)}\label{fig:NonCommutationRescaling_b}
        \vspace{-1.2ex}
        \includegraphics[width=0.95\textwidth]{NonCommutation/QARescale.png}
    \end{subfigure}
    \caption[Effective energy rescaling factor for non-commuting dynamics]{Deviation of the effective energy rescaling factor $1+x_0\langle\sigma_z\rangle(t)$ from the ideal QND value, as a function of annealing duration $T$ and meter frequency $\omega$. (a) Landau-Zener qubit. (b) 3-qubit random Ising chain. The universally negative values indicate that non-commuting meter dynamics ($\omega > 0$) strictly reduce the speedup, consistent with the bound in Eq.~\eqref{eq:speedup_bound}. The fast oscillatory structure at small $\omega$ reflects the Rabi oscillation term in Eq.~\eqref{eq:sigma_z_final}.}
    \label{fig:NonCommutationRescaling}
\end{figure}

\qquad Fig.~\ref{fig:NonCommutationRescaling} displays the deviation of the effective rescaling factor $1 + x_0\langle\sigma_z\rangle(t)$ from its ideal value $(1+x_0) = 2$ as a function of annealing time $T$ and meter frequency $\omega$. Several features confirm the analytical predictions. The deviation is universally negative for all $\omega > 0$, consistent with the bound in Eq.~\eqref{eq:speedup_bound}. Fast oscillations are visible at small $\omega$, corresponding to the Rabi oscillation term in Eq.~\eqref{eq:sigma_z_final}. The oscillations decay near the avoided crossing, where $E_n(t) \to E_{\min}$ and the Rabi frequency $\Omega_n$ drops, increasing the oscillation period beyond the local time window. The rescaling deficit grows monotonically with $\omega$.

\subsection{Interference Patterns in the Fidelity Landscape}

\qquad Fig.~\ref{fig:interference_analysis} provides a direct test of the interference factor derived in Section~\ref{sec:interference}. Three panels are presented. The fidelity difference $\Delta F = F(T, \omega) - F(T, 0)$ reveals characteristic ripple modulations at small $\omega$ and short $T$. These ripples are superimposed on the monotonic fidelity decrease predicted by the time-averaged analysis. The analytical interference factor $C(T, \omega)$ from Eq.~\eqref{eq:interference_factor_general}, plotted with its zero-crossing contours (black dashed lines). The contours are defined by $2x_0\int_0^T \Omega_{\text{eff}}(s)\,ds = (2n+1)\pi/2$. The full fidelity map overlaid with the $C = 0$ contours (white dashed lines). The zero-crossings of the analytical interference factor align precisely with the numerical fidelity ripples, confirming that the oscillatory structure arises from coherent interference between the two evolution branches.

\begin{figure}[htbp]
    \centering
    \begin{subfigure}[t]{0.32\textwidth}
        \centering
        \refstepcounter{subfigure}\makebox[\textwidth][l]{\hspace{-2mm}(\thesubfigure)}\label{fig:interference_analysis_a}
        \vspace{-1.2ex}
        \includegraphics[width=\textwidth]{NonCommutation/WNonFidDiff.png}
    \end{subfigure}%
    \hfill
    \begin{subfigure}[t]{0.32\textwidth}
        \centering
        \refstepcounter{subfigure}\makebox[\textwidth][l]{\hspace{-2mm}(\thesubfigure)}\label{fig:interference_analysis_b}
        \vspace{-1.2ex}
        \includegraphics[width=\textwidth]{NonCommutation/InterferenceFactor.png}
    \end{subfigure}%
    \hfill
    \begin{subfigure}[t]{0.32\textwidth}
        \centering
        \refstepcounter{subfigure}\makebox[\textwidth][l]{\hspace{-2mm}(\thesubfigure)}\label{fig:interference_analysis_c}
        \vspace{-1.2ex}
        \includegraphics[width=\textwidth]{NonCommutation/WNonCommuting.png}
    \end{subfigure}
    \caption[Interference effects in non-commuting meter regime]{Interference effects in the non-commuting meter regime for the Landau-Zener model with $x_0 = 1$. (a) Fidelity difference $\Delta F = F(T, \omega) - F(T, 0)$ showing ripple modulations superimposed on the monotonic decrease. (b) Analytical interference factor $C(T, \omega)$ from Eq.~\eqref{eq:interference_factor_general}, with zero-crossing contours (black dashed). (c) Full fidelity map with $C = 0$ contours overlaid (white dashed), demonstrating quantitative agreement between the analytical prediction and numerical ripple positions.}
    \label{fig:interference_analysis}
\end{figure}

\qquad The physical interpretation is as follows. At small $\omega$, the Rabi oscillations of the meter are slow enough that they are not averaged out over the annealing time $T$. The cosine term in Eq.~\eqref{eq:sigma_z_final} alternates between $+1$ and $-1$, periodically enhancing and suppressing the effective dephasing. As $\omega$ increases, the oscillation frequency grows and the oscillatory contribution averages away, leaving only the monotonic suppression from the time-averaged term. The transition between these regimes is governed by the condition $\Omega_n T \sim 1$.