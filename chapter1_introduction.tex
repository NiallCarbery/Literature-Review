% Chapter 1: Introduction
\newpage
\pagenumbering{arabic}
\doublespacing
\chapter{Introduction}\label{sec:introduction}

\qquad Quantum annealing protocols aim to generate optimization solutions to problems, typically combinatorial, by encoding the solution to the problem as the ground state of a Hamiltonian. The system is initialized in the ground state of a simple Hamiltonian $H_0$, and through adiabatic evolution, is transformed to the ground state of a problem Hamiltonian $H_P$ that encodes the optimization landscape. This adiabatic evolution is in principle a slowly varying parameter of a quantum system that is changing sufficiently slowly i.e. an electromagnetic field, such that we remain the in the ground state of the quantum system, thus a ground state of $H_p$ and the solution to our problem\cite{AlbashLidar2018} The time-dependent Hamiltonian takes the form
%%
\begin{equation}\label{eq:annealing_hamiltonian}
    H(t) = (1-\lambda(t))H_0 + \lambda(t)H_P.
\end{equation}
%%
\qquad Quantum annealing, however, is a heuristic method with a non-quantified speed-up compared to classical optimization techniques. The fundamental limitation arises from the \textit{adiabatic theorem}, which requires evolution times much longer than the minimum energy gap between the ground and first excited states. For many combinatorial problems, this gap decreases exponentially with system size, leading to impractical computation times that negate any potential quantum advantage. Despite this annealing devices represent a state of the art way of studying time evolutions of this form better than classical computers,\cite{DWAVEBeyondClassical2025} and offer a scaling benefit frustrated models over simulated annealing.\cite{AlbashQASpeedUp2018}

\qquad A key element of some of these shortcuts to adiabaticity rely on knowing the eigenspectrum of the Hamiltonian prior to the annealing protocol. For the application to quantum annealing, knowing the eigenspectrum \textit{a priori} defeats the purpose of the protocol, as if one could efficiently compute the eigenspectrum, the optimization problem would already be solved. \cite{ControlReviewCampbell} Therefore, there is a trade-off between practical applicability and performance. 

\qquad The thesis will be structured in four main sections. First, in Ch. \ref{Ch1:TheoreticalFoundations} we will review the theoretical foundations of quantum annealing, including the formulation of optimization problems as Hamiltonians and the adiabatic theorem. Then in Ch. \ref{Ch2:SpeedingUpQuantumAnnealing} we will introduce a degree of freedom and dephasing techniques developed in \textit{Speeding Up Quantum Annealing with Engineered Dephasing}\cite{DephasingPaper}. From there we will test the implementability of these techniques for current hardware and their performance. This will involve modifying the connection of the degree of freedom to the system in Ch. \ref{Ch3:ModifyingConnection} and modifying its interaction terms in Ch. \ref{Ch4:ModifyingInteraction}. Finally, we will compare these techniques to other developments in the field of quantum optimal control and shortcuts to adiabaticity (STA) that have been developed in the context of quantum annealing.